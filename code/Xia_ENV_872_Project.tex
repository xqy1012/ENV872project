\documentclass[12pt,]{article}
\usepackage{lmodern}
\usepackage{amssymb,amsmath}
\usepackage{ifxetex,ifluatex}
\usepackage{fixltx2e} % provides \textsubscript
\ifnum 0\ifxetex 1\fi\ifluatex 1\fi=0 % if pdftex
  \usepackage[T1]{fontenc}
  \usepackage[utf8]{inputenc}
\else % if luatex or xelatex
  \ifxetex
    \usepackage{mathspec}
  \else
    \usepackage{fontspec}
  \fi
  \defaultfontfeatures{Ligatures=TeX,Scale=MatchLowercase}
    \setmainfont[]{Times New Roman}
\fi
% use upquote if available, for straight quotes in verbatim environments
\IfFileExists{upquote.sty}{\usepackage{upquote}}{}
% use microtype if available
\IfFileExists{microtype.sty}{%
\usepackage{microtype}
\UseMicrotypeSet[protrusion]{basicmath} % disable protrusion for tt fonts
}{}
\usepackage[margin=2.54cm]{geometry}
\usepackage{hyperref}
\hypersetup{unicode=true,
            pdftitle={2017, 2018 Ozone in North Carolina},
            pdfauthor={Qianyi Xia},
            pdfborder={0 0 0},
            breaklinks=true}
\urlstyle{same}  % don't use monospace font for urls
\usepackage{color}
\usepackage{fancyvrb}
\newcommand{\VerbBar}{|}
\newcommand{\VERB}{\Verb[commandchars=\\\{\}]}
\DefineVerbatimEnvironment{Highlighting}{Verbatim}{commandchars=\\\{\}}
% Add ',fontsize=\small' for more characters per line
\usepackage{framed}
\definecolor{shadecolor}{RGB}{248,248,248}
\newenvironment{Shaded}{\begin{snugshade}}{\end{snugshade}}
\newcommand{\KeywordTok}[1]{\textcolor[rgb]{0.13,0.29,0.53}{\textbf{#1}}}
\newcommand{\DataTypeTok}[1]{\textcolor[rgb]{0.13,0.29,0.53}{#1}}
\newcommand{\DecValTok}[1]{\textcolor[rgb]{0.00,0.00,0.81}{#1}}
\newcommand{\BaseNTok}[1]{\textcolor[rgb]{0.00,0.00,0.81}{#1}}
\newcommand{\FloatTok}[1]{\textcolor[rgb]{0.00,0.00,0.81}{#1}}
\newcommand{\ConstantTok}[1]{\textcolor[rgb]{0.00,0.00,0.00}{#1}}
\newcommand{\CharTok}[1]{\textcolor[rgb]{0.31,0.60,0.02}{#1}}
\newcommand{\SpecialCharTok}[1]{\textcolor[rgb]{0.00,0.00,0.00}{#1}}
\newcommand{\StringTok}[1]{\textcolor[rgb]{0.31,0.60,0.02}{#1}}
\newcommand{\VerbatimStringTok}[1]{\textcolor[rgb]{0.31,0.60,0.02}{#1}}
\newcommand{\SpecialStringTok}[1]{\textcolor[rgb]{0.31,0.60,0.02}{#1}}
\newcommand{\ImportTok}[1]{#1}
\newcommand{\CommentTok}[1]{\textcolor[rgb]{0.56,0.35,0.01}{\textit{#1}}}
\newcommand{\DocumentationTok}[1]{\textcolor[rgb]{0.56,0.35,0.01}{\textbf{\textit{#1}}}}
\newcommand{\AnnotationTok}[1]{\textcolor[rgb]{0.56,0.35,0.01}{\textbf{\textit{#1}}}}
\newcommand{\CommentVarTok}[1]{\textcolor[rgb]{0.56,0.35,0.01}{\textbf{\textit{#1}}}}
\newcommand{\OtherTok}[1]{\textcolor[rgb]{0.56,0.35,0.01}{#1}}
\newcommand{\FunctionTok}[1]{\textcolor[rgb]{0.00,0.00,0.00}{#1}}
\newcommand{\VariableTok}[1]{\textcolor[rgb]{0.00,0.00,0.00}{#1}}
\newcommand{\ControlFlowTok}[1]{\textcolor[rgb]{0.13,0.29,0.53}{\textbf{#1}}}
\newcommand{\OperatorTok}[1]{\textcolor[rgb]{0.81,0.36,0.00}{\textbf{#1}}}
\newcommand{\BuiltInTok}[1]{#1}
\newcommand{\ExtensionTok}[1]{#1}
\newcommand{\PreprocessorTok}[1]{\textcolor[rgb]{0.56,0.35,0.01}{\textit{#1}}}
\newcommand{\AttributeTok}[1]{\textcolor[rgb]{0.77,0.63,0.00}{#1}}
\newcommand{\RegionMarkerTok}[1]{#1}
\newcommand{\InformationTok}[1]{\textcolor[rgb]{0.56,0.35,0.01}{\textbf{\textit{#1}}}}
\newcommand{\WarningTok}[1]{\textcolor[rgb]{0.56,0.35,0.01}{\textbf{\textit{#1}}}}
\newcommand{\AlertTok}[1]{\textcolor[rgb]{0.94,0.16,0.16}{#1}}
\newcommand{\ErrorTok}[1]{\textcolor[rgb]{0.64,0.00,0.00}{\textbf{#1}}}
\newcommand{\NormalTok}[1]{#1}
\usepackage{longtable,booktabs}
\usepackage{graphicx,grffile}
\makeatletter
\def\maxwidth{\ifdim\Gin@nat@width>\linewidth\linewidth\else\Gin@nat@width\fi}
\def\maxheight{\ifdim\Gin@nat@height>\textheight\textheight\else\Gin@nat@height\fi}
\makeatother
% Scale images if necessary, so that they will not overflow the page
% margins by default, and it is still possible to overwrite the defaults
% using explicit options in \includegraphics[width, height, ...]{}
\setkeys{Gin}{width=\maxwidth,height=\maxheight,keepaspectratio}
\IfFileExists{parskip.sty}{%
\usepackage{parskip}
}{% else
\setlength{\parindent}{0pt}
\setlength{\parskip}{6pt plus 2pt minus 1pt}
}
\setlength{\emergencystretch}{3em}  % prevent overfull lines
\providecommand{\tightlist}{%
  \setlength{\itemsep}{0pt}\setlength{\parskip}{0pt}}
\setcounter{secnumdepth}{5}
% Redefines (sub)paragraphs to behave more like sections
\ifx\paragraph\undefined\else
\let\oldparagraph\paragraph
\renewcommand{\paragraph}[1]{\oldparagraph{#1}\mbox{}}
\fi
\ifx\subparagraph\undefined\else
\let\oldsubparagraph\subparagraph
\renewcommand{\subparagraph}[1]{\oldsubparagraph{#1}\mbox{}}
\fi

%%% Use protect on footnotes to avoid problems with footnotes in titles
\let\rmarkdownfootnote\footnote%
\def\footnote{\protect\rmarkdownfootnote}

%%% Change title format to be more compact
\usepackage{titling}

% Create subtitle command for use in maketitle
\providecommand{\subtitle}[1]{
  \posttitle{
    \begin{center}\large#1\end{center}
    }
}

\setlength{\droptitle}{-2em}

  \title{2017, 2018 Ozone in North Carolina}
    \pretitle{\vspace{\droptitle}\centering\huge}
  \posttitle{\par}
  \subtitle{\url{https://github.com/xqy1012/ENV872project}}
  \author{Qianyi Xia}
    \preauthor{\centering\large\emph}
  \postauthor{\par}
    \date{}
    \predate{}\postdate{}
  

\begin{document}
\maketitle
\begin{abstract}
Experimental overview. This section should be no longer than 250 words.
\end{abstract}

\newpage

\tableofcontents  \newpage
\listoftables  \newpage
\listoffigures  \newpage

\section{Research Question and
Rationale}\label{research-question-and-rationale}

\begin{itemize}
\item[]      According to American Lung Association, the 2018 "State of the Air" report reveals that unhealthful levels of pollution put the citizens at risk.  Compare to 2017 repost, North Carolina Ozone Pollution worsened in 2018 compare to 2017 because there are more unhealthy days of high ozone in 2018's year report, especially in some cities. The report indicated that more work needs to be done to protect the health of residents from harms of ozone pollution. However, the EPA website showed that in Charlotte, NC, the number of days reaching unhealthy for sensitive groups for ozone pollution has been continue decreasing to 2017. There is no analysis report for ozone pollution in 2018 from EPA yet. Therefore, the interest for this project is to verify the accuracy of the news report by American Lung Association and analyze the ozone pollution in 2017 and 2018 in North Carolina.  
\item[]     It is important to study ozone because human ozone exposure may result in adverse health effects including reduced lung function, respiratory symptoms, asthma, and other premature mortality from respiratory causes. For the nature, ozone damages vegetation, decreasing crop yields, and even may alter ecosystem structure. Ozone is also a greenhouse gas that contribute to global warming.  
\item[]      There will be three main research questions for the report: 1. Does year 2018 has a worsened ozone pollution than 2017? 2. Are there any trend /fluctuation of ozone pollution among different months within a year? Is it true in this data that ozone level is higher when it is hot, dry and sunny in the summer? Does the trend similar in year 2017 and year 2018? 3. Does Ozone pollution level related to population size/ household income/ population density of a county? I will use the raw data that downloaded from EPA our door air quality dataset for 2017 and 2018, combining with demographic data obtained from United States Census Bureau website.
\end{itemize}

\newpage

\section{Dataset Information}\label{dataset-information}

\newpage

\section{Exploratory Data Analysis and
Wrangling}\label{exploratory-data-analysis-and-wrangling}

 First, display the data summary to get information of the data on: the
dimension of data, how many different counties/ Ozone monitoring sites
are there in North Carolina, where does the county located
(e.g.~Raleigh, Charlotte, etc.), which columns of the raw data are
useful in this report analysis, filter the useful column and save as new
files. Also check the format of concentration, AQI, and date, check on
if these value needs reformat to numeric or dates. Combine the 2017,
2018 data together and save in another file.

\begin{Shaded}
\begin{Highlighting}[]
\KeywordTok{dim}\NormalTok{(EPA_Ozone_}\FloatTok{2017.}\NormalTok{data)}
\end{Highlighting}
\end{Shaded}

\begin{verbatim}
## [1] 10219    20
\end{verbatim}

\begin{Shaded}
\begin{Highlighting}[]
\KeywordTok{colnames}\NormalTok{(EPA_Ozone_}\FloatTok{2017.}\NormalTok{data)}
\end{Highlighting}
\end{Shaded}

\begin{verbatim}
##  [1] "Date"                                
##  [2] "Source"                              
##  [3] "Site.ID"                             
##  [4] "POC"                                 
##  [5] "Daily.Max.8.hour.Ozone.Concentration"
##  [6] "UNITS"                               
##  [7] "DAILY_AQI_VALUE"                     
##  [8] "Site.Name"                           
##  [9] "DAILY_OBS_COUNT"                     
## [10] "PERCENT_COMPLETE"                    
## [11] "AQS_PARAMETER_CODE"                  
## [12] "AQS_PARAMETER_DESC"                  
## [13] "CBSA_CODE"                           
## [14] "CBSA_NAME"                           
## [15] "STATE_CODE"                          
## [16] "STATE"                               
## [17] "COUNTY_CODE"                         
## [18] "COUNTY"                              
## [19] "SITE_LATITUDE"                       
## [20] "SITE_LONGITUDE"
\end{verbatim}

\begin{Shaded}
\begin{Highlighting}[]
\KeywordTok{summary}\NormalTok{(EPA_Ozone_}\FloatTok{2017.}\NormalTok{data}\OperatorTok{$}\NormalTok{Site.Name)}
\end{Highlighting}
\end{Shaded}

\begin{verbatim}
##                                                                  
##                                                              206 
##                                                         Beaufort 
##                                                              338 
##                                                       Bent Creek 
##                                                              240 
##                                                     Bethany sch. 
##                                                              240 
##                                                       Blackstone 
##                                                              355 
##                                                      Bryson City 
##                                                              223 
##                                                       Bushy Fork 
##                                                              241 
##                                                           Butner 
##                                                              243 
##                                                           Candor 
##                                                              325 
##                                                     Castle Hayne 
##                                                              239 
##                                                     Cherry Grove 
##                                                              232 
##                                                  Clemmons Middle 
##                                                              240 
##                                                          Coweeta 
##                                                              344 
##                                                        Cranberry 
##                                                              307 
##                                                           Crouse 
##                                                              237 
##                                                    Durham Armory 
##                                                              245 
##                                              Frying Pan Mountain 
##                                                              229 
##                                             Garinger High School 
##                                                              358 
##                                                    Hattie Avenue 
##                                                              242 
##                                                 Honeycutt School 
##                                                              219 
##                                                Jamesville School 
##                                                              244 
##                                                      Joanna Bald 
##                                                              227 
##                                                          Leggett 
##                                                              236 
##                                                    Lenoir (city) 
##                                                              239 
##                                           Lenoir Co. Comm. Coll. 
##                                                              244 
##                                                   Linville Falls 
##                                                              234 
##                                                Mendenhall School 
##                                                              239 
##                                                 Millbrook School 
##                                                              339 
##                                                    Monroe School 
##                                                              236 
##                                                     Mt. Mitchell 
##                                                              199 
## OZONE MONITOR ON SW SIDE OF TOWER/MET EQUIPMENT 10FT ABOVE TOWER 
##                                                              199 
##                                                Pitt Agri. Center 
##                                                              245 
##                                                    Purchase Knob 
##                                                              234 
##                                                         Rockwell 
##                                                              354 
##                                            Taylorsville Liledoun 
##                                                              234 
##                                                      Union Cross 
##                                                              243 
##                                               University Meadows 
##                                                              243 
##                                                             Wade 
##                                                              245 
##                                               Waynesville School 
##                                                              237 
##                                                West Johnston Co. 
##                                                              245
\end{verbatim}

\begin{Shaded}
\begin{Highlighting}[]
\KeywordTok{summary}\NormalTok{(EPA_Ozone_}\FloatTok{2017.}\NormalTok{data}\OperatorTok{$}\NormalTok{COUNTY)}
\end{Highlighting}
\end{Shaded}

\begin{verbatim}
##   Alexander       Avery    Buncombe    Caldwell    Carteret     Caswell 
##         234         541         240         239         338         232 
##  Cumberland      Durham   Edgecombe     Forsyth      Graham   Granville 
##         464         245         236         725         227         243 
##    Guilford     Haywood     Jackson    Johnston         Lee      Lenoir 
##         239         700         199         245         355         244 
##     Lincoln       Macon      Martin Mecklenburg  Montgomery New Hanover 
##         237         344         244         601         325         239 
##      Person        Pitt  Rockingham       Rowan       Swain       Union 
##         241         245         240         354         429         236 
##        Wake      Yancey 
##         339         199
\end{verbatim}

\begin{Shaded}
\begin{Highlighting}[]
\KeywordTok{class}\NormalTok{(EPA_Ozone_}\FloatTok{2017.}\NormalTok{data}\OperatorTok{$}\NormalTok{Daily.Max.}\FloatTok{8.}\NormalTok{hour.Ozone.Concentration)}
\end{Highlighting}
\end{Shaded}

\begin{verbatim}
## [1] "numeric"
\end{verbatim}

\begin{Shaded}
\begin{Highlighting}[]
\KeywordTok{class}\NormalTok{(EPA_Ozone_}\FloatTok{2017.}\NormalTok{data}\OperatorTok{$}\NormalTok{DAILY_AQI_VALUE)}
\end{Highlighting}
\end{Shaded}

\begin{verbatim}
## [1] "integer"
\end{verbatim}

\begin{Shaded}
\begin{Highlighting}[]
\KeywordTok{class}\NormalTok{(EPA_Ozone_}\FloatTok{2017.}\NormalTok{data}\OperatorTok{$}\NormalTok{Date)}
\end{Highlighting}
\end{Shaded}

\begin{verbatim}
## [1] "factor"
\end{verbatim}

\pagebreak
The summary table include the information of ozone mean, min, max AQI,
and mean concentration grouped by year and counties.

\begin{longtable}[]{@{}rlrrlrr@{}}
\caption{Summary of Ozone AQI/Concentration in year 2017 and 2018 by
counties}\tabularnewline
\toprule
year & COUNTY & MeanOzoneAQI & MeanOzoneConc & Units & minOzoneAQI &
maxOzoneAQI\tabularnewline
\midrule
\endfirsthead
\toprule
year & COUNTY & MeanOzoneAQI & MeanOzoneConc & Units & minOzoneAQI &
maxOzoneAQI\tabularnewline
\midrule
\endhead
2017 & Alexander & 40.19231 & 0.0426239 & ppm & 14 & 93\tabularnewline
2017 & Avery & 39.42884 & 0.0419316 & ppm & 16 & 93\tabularnewline
2017 & Buncombe & 38.90833 & 0.0414750 & ppm & 12 & 84\tabularnewline
2017 & Caldwell & 42.23431 & 0.0442427 & ppm & 15 & 90\tabularnewline
2017 & Carteret & 35.25148 & 0.0379497 & ppm & 13 & 64\tabularnewline
2017 & Caswell & 38.68103 & 0.0414009 & ppm & 13 & 71\tabularnewline
2017 & Cumberland & 40.98922 & 0.0432220 & ppm & 10 & 100\tabularnewline
2017 & Durham & 40.04898 & 0.0423306 & ppm & 6 & 100\tabularnewline
2017 & Edgecombe & 38.77966 & 0.0413347 & ppm & 11 & 80\tabularnewline
2017 & Forsyth & 44.03034 & 0.0456745 & ppm & 15 & 100\tabularnewline
2017 & Graham & 41.14978 & 0.0436432 & ppm & 12 & 87\tabularnewline
2017 & Granville & 42.16872 & 0.0441893 & ppm & 19 & 93\tabularnewline
2017 & Guilford & 44.06276 & 0.0454812 & ppm & 16 & 112\tabularnewline
2017 & Haywood & 42.37714 & 0.0448114 & ppm & 13 & 84\tabularnewline
2017 & Jackson & 45.34171 & 0.0467889 & ppm & 16 & 100\tabularnewline
2017 & Johnston & 38.42449 & 0.0408857 & ppm & 11 & 93\tabularnewline
2017 & Lee & 37.66479 & 0.0402225 & ppm & 8 & 97\tabularnewline
2017 & Lenoir & 38.97541 & 0.0412254 & ppm & 11 & 90\tabularnewline
2017 & Lincoln & 41.55696 & 0.0437764 & ppm & 14 & 93\tabularnewline
2017 & Macon & 33.51744 & 0.0359070 & ppm & 8 & 74\tabularnewline
2017 & Martin & 37.67213 & 0.0402910 & ppm & 16 & 74\tabularnewline
2017 & Mecklenburg & 41.10815 & 0.0425757 & ppm & 7 & 115\tabularnewline
2017 & Montgomery & 36.59077 & 0.0392338 & ppm & 10 & 71\tabularnewline
2017 & New Hanover & 35.50628 & 0.0380837 & ppm & 5 & 67\tabularnewline
2017 & Person & 40.46888 & 0.0429793 & ppm & 12 & 74\tabularnewline
2017 & Pitt & 39.38776 & 0.0417469 & ppm & 16 & 87\tabularnewline
2017 & Rockingham & 41.43750 & 0.0436375 & ppm & 15 & 84\tabularnewline
2017 & Rowan & 37.71469 & 0.0400876 & ppm & 11 & 80\tabularnewline
2017 & Swain & 35.13054 & 0.0378228 & ppm & 8 & 71\tabularnewline
2017 & Union & 42.52119 & 0.0441441 & ppm & 15 & 115\tabularnewline
2017 & Wake & 37.95870 & 0.0399204 & ppm & 7 & 100\tabularnewline
2017 & Yancey & 45.37688 & 0.0472161 & ppm & 18 & 93\tabularnewline
2018 & Alexander & 38.74737 & 0.0405193 & ppm & 16 & 100\tabularnewline
2018 & Avery & 38.35237 & 0.0404780 & ppm & 0 & 87\tabularnewline
2018 & Buncombe & 37.46786 & 0.0395464 & ppm & 12 & 93\tabularnewline
2018 & Caldwell & 39.22300 & 0.0409233 & ppm & 10 & 93\tabularnewline
2018 & Carteret & 37.96861 & 0.0404081 & ppm & 17 & 77\tabularnewline
2018 & Caswell & 39.30980 & 0.0409216 & ppm & 10 & 97\tabularnewline
2018 & Cumberland & 41.69593 & 0.0433640 & ppm & 17 & 100\tabularnewline
2018 & Durham & 38.05155 & 0.0400241 & ppm & 9 & 100\tabularnewline
2018 & Edgecombe & 38.57708 & 0.0406680 & ppm & 15 & 84\tabularnewline
2018 & Forsyth & 43.70955 & 0.0446578 & ppm & 12 & 101\tabularnewline
2018 & Graham & 42.35599 & 0.0440162 & ppm & 19 & 97\tabularnewline
2018 & Granville & 40.03833 & 0.0416585 & ppm & 10 & 93\tabularnewline
2018 & Guilford & 43.80989 & 0.0444867 & ppm & 13 & 105\tabularnewline
2018 & Haywood & 41.40159 & 0.0432514 & ppm & 0 & 105\tabularnewline
2018 & Johnston & 37.75839 & 0.0397114 & ppm & 15 & 90\tabularnewline
2018 & Lee & 39.83256 & 0.0420884 & ppm & 17 & 77\tabularnewline
2018 & Lenoir & 39.94902 & 0.0419373 & ppm & 16 & 108\tabularnewline
2018 & Lincoln & 39.78113 & 0.0417849 & ppm & 13 & 90\tabularnewline
2018 & Macon & 32.52647 & 0.0346029 & ppm & 12 & 93\tabularnewline
2018 & Martin & 38.86192 & 0.0409707 & ppm & 6 & 80\tabularnewline
2018 & Mecklenburg & 40.81804 & 0.0415633 & ppm & 13 &
108\tabularnewline
2018 & Montgomery & 34.46291 & 0.0369110 & ppm & 0 & 71\tabularnewline
2018 & New Hanover & 37.72199 & 0.0398423 & ppm & 16 & 77\tabularnewline
2018 & Person & 38.85091 & 0.0408655 & ppm & 10 & 97\tabularnewline
2018 & Pitt & 39.36585 & 0.0413031 & ppm & 16 & 87\tabularnewline
2018 & Rockingham & 38.50904 & 0.0404187 & ppm & 10 & 90\tabularnewline
2018 & Rowan & 36.34277 & 0.0389057 & ppm & 14 & 74\tabularnewline
2018 & Swain & 36.08725 & 0.0383870 & ppm & 5 & 80\tabularnewline
2018 & Union & 42.85433 & 0.0435591 & ppm & 16 & 122\tabularnewline
2018 & Wake & 38.01183 & 0.0398136 & ppm & 6 & 90\tabularnewline
2018 & Yancey & 42.90076 & 0.0447786 & ppm & 4 & 84\tabularnewline
\bottomrule
\end{longtable}

\pagebreak

\begin{itemize}
\item[]   Then for exploratory graphs, normality is visualized first by QQ norm plots in Figure 1.The sample size is larger than 5,000, so Kolmogorov-Smirnov test is applied together to test normality for ozone daily AQI value. The result shows that both original data and log-transformed data are not normally distributed. \
\item[] Correlation between AQI and Ozone 8-hour concentration is also test with spearman correlation and plotted in Figure 2. In the raw data, there are both AQI and Ozone 8-hour max concentrations, the reason for correlation test is to make sure if using AQI is appropriate to represent Ozone pollution level. The correlation figure shows that there is a correlation coefficient of 1.00 between these two variables, so only AQI will be used in later analysis.   \item[] Figure 3 displays a monthly boxplot for year 2017 and 2018, February and September have higher ozone AQI in 2017, it is hard to identify other obvious differences only from this figure. \
\item[] Figure 4 shows a LOESS trend in 2017 and 2018, it provide a first impression on the potential seasonal trend before conducting statistical analysis in the next section. \
\item[] There is also a map generated to show different location of the counties, how they distributed in NC.
\end{itemize}

\begin{figure}
\centering
\includegraphics{Xia_ENV_872_Project_files/figure-latex/exploration 1-1.pdf}
\caption{QQ plots for 2017 and 2018 with/without log transform}
\end{figure}

\begin{figure}
\centering
\includegraphics{Xia_ENV_872_Project_files/figure-latex/exploration 2-1.pdf}
\caption{Correlation Between AQI and 8-hour Max Ozone Concentration}
\end{figure}

\begin{figure}
\centering
\includegraphics{Xia_ENV_872_Project_files/figure-latex/exploration 3-1.pdf}
\caption{Monthly Box Plot of 2017 and 2018 Ozone AQI}
\end{figure}

\begin{figure}
\centering
\includegraphics{Xia_ENV_872_Project_files/figure-latex/exploration 5-1.pdf}
\caption{2017 and 2018 ozone AQI change through out year}
\end{figure}

\newpage

\section{Analysis}\label{analysis}

\begin{itemize}
\item[] First analysis is the Wilcoxon test to test if ozone AQI in 2017 is lower than 2018. Wilcoxon test is conducted since the data are non-parametric data, which means not normally distributed. T-test or ANOVA test cannot be used here. Both the difference for yearly mean and monthly difference are tested. One interesting finding before conducting the test is that the 2017 AQI mean is actually higher than the 2018 AQI, so the null hypothesis is adjusted as follow:  
\item[]H0: 2017 and 2018 mean ozone AQI are identical.  \
\item[]Ha: 2017 has a higher mean ozone AQI than 2018.  \
\item[]From the result and the visualization figures we see that, the 2017 has a significantly lower mean AQI than 2018 (p-value <0.001). Among the 12 months, February (p-value <0.001), May (p-value <0.001), August (p-value <0.001), September (p-value <0.001), October (p-value = 0.04), November (p-value <0.001), December (p-value <0.001) also have higher mean ozone AQI than 2018.
\item[]While in January, April, June, the p-value is 1, and the mean of AQI in 2018 in these month are higher than 2017.
\end{itemize}

\begin{Shaded}
\begin{Highlighting}[]
\CommentTok{#ANOVA Test Assumption}
\CommentTok{#Previous tested for normality not met}
\KeywordTok{ks.test}\NormalTok{(EPA_Ozone_}\FloatTok{2017.}\NormalTok{data.Processed}\OperatorTok{$}\NormalTok{DAILY_AQI_VALUE,pnorm)}
\end{Highlighting}
\end{Shaded}

\begin{verbatim}
## Warning in ks.test(EPA_Ozone_2017.data.Processed$DAILY_AQI_VALUE, pnorm):
## ties should not be present for the Kolmogorov-Smirnov test
\end{verbatim}

\begin{verbatim}
## 
##  One-sample Kolmogorov-Smirnov test
## 
## data:  EPA_Ozone_2017.data.Processed$DAILY_AQI_VALUE
## D = 1, p-value < 2.2e-16
## alternative hypothesis: two-sided
\end{verbatim}

\begin{Shaded}
\begin{Highlighting}[]
\KeywordTok{ks.test}\NormalTok{(}\KeywordTok{log}\NormalTok{(EPA_Ozone_}\FloatTok{2017.}\NormalTok{data.Processed}\OperatorTok{$}\NormalTok{DAILY_AQI_VALUE),pnorm)}
\end{Highlighting}
\end{Shaded}

\begin{verbatim}
## Warning in ks.test(log(EPA_Ozone_2017.data.Processed$DAILY_AQI_VALUE),
## pnorm): ties should not be present for the Kolmogorov-Smirnov test
\end{verbatim}

\begin{verbatim}
## 
##  One-sample Kolmogorov-Smirnov test
## 
## data:  log(EPA_Ozone_2017.data.Processed$DAILY_AQI_VALUE)
## D = 0.99173, p-value < 2.2e-16
## alternative hypothesis: two-sided
\end{verbatim}

\begin{Shaded}
\begin{Highlighting}[]
\KeywordTok{ks.test}\NormalTok{(EPA_Ozone_}\FloatTok{2018.}\NormalTok{data.Processed}\OperatorTok{$}\NormalTok{DAILY_AQI_VALUE,pnorm)}
\end{Highlighting}
\end{Shaded}

\begin{verbatim}
## Warning in ks.test(EPA_Ozone_2018.data.Processed$DAILY_AQI_VALUE, pnorm):
## ties should not be present for the Kolmogorov-Smirnov test
\end{verbatim}

\begin{verbatim}
## 
##  One-sample Kolmogorov-Smirnov test
## 
## data:  EPA_Ozone_2018.data.Processed$DAILY_AQI_VALUE
## D = 0.99821, p-value < 2.2e-16
## alternative hypothesis: two-sided
\end{verbatim}

\begin{Shaded}
\begin{Highlighting}[]
\KeywordTok{ks.test}\NormalTok{(}\KeywordTok{log}\NormalTok{(EPA_Ozone_}\FloatTok{2018.}\NormalTok{data.Processed}\OperatorTok{$}\NormalTok{DAILY_AQI_VALUE),pnorm)}
\end{Highlighting}
\end{Shaded}

\begin{verbatim}
## Warning in ks.test(log(EPA_Ozone_2018.data.Processed$DAILY_AQI_VALUE),
## pnorm): ties should not be present for the Kolmogorov-Smirnov test
\end{verbatim}

\begin{verbatim}
## 
##  One-sample Kolmogorov-Smirnov test
## 
## data:  log(EPA_Ozone_2018.data.Processed$DAILY_AQI_VALUE)
## D = 0.98881, p-value < 2.2e-16
## alternative hypothesis: two-sided
\end{verbatim}

\begin{Shaded}
\begin{Highlighting}[]
\CommentTok{#test for homogenecity of variance met}
\KeywordTok{class}\NormalTok{(EPA_totalOzone.data.processed}\OperatorTok{$}\NormalTok{year)}
\end{Highlighting}
\end{Shaded}

\begin{verbatim}
## [1] "numeric"
\end{verbatim}

\begin{Shaded}
\begin{Highlighting}[]
\NormalTok{EPA_totalOzone.data.processed}\OperatorTok{$}\NormalTok{year <-}\StringTok{ }\KeywordTok{as.factor}\NormalTok{(EPA_totalOzone.data.processed}\OperatorTok{$}\NormalTok{year)}

\KeywordTok{sd}\NormalTok{(EPA_Ozone_}\FloatTok{2017.}\NormalTok{data.Processed}\OperatorTok{$}\NormalTok{DAILY_AQI_VALUE)}\OperatorTok{/}
\StringTok{  }\KeywordTok{sd}\NormalTok{(EPA_Ozone_}\FloatTok{2018.}\NormalTok{data.Processed}\OperatorTok{$}\NormalTok{DAILY_AQI_VALUE)}
\end{Highlighting}
\end{Shaded}

\begin{verbatim}
## [1] 0.87895
\end{verbatim}

\begin{Shaded}
\begin{Highlighting}[]
\CommentTok{# bartlett.test(EPA_totalOzone.data.processed$DAILY_AQI_VALUE~}
\CommentTok{# EPA_totalOzone.data.processed$year, EPA_totalOzone.data.processed)}
     
\CommentTok{#T-test Assumption not met, data is not normally distributed. }
\CommentTok{#So I will conduct Mann-Whitney-Wilcoxon Test.}

\KeywordTok{mean}\NormalTok{(EPA_Ozone_}\FloatTok{2017.}\NormalTok{data.Processed}\OperatorTok{$}\NormalTok{DAILY_AQI_VALUE)}
\end{Highlighting}
\end{Shaded}

\begin{verbatim}
## [1] 39.86897
\end{verbatim}

\begin{Shaded}
\begin{Highlighting}[]
\KeywordTok{mean}\NormalTok{(EPA_Ozone_}\FloatTok{2018.}\NormalTok{data.Processed}\OperatorTok{$}\NormalTok{DAILY_AQI_VALUE)}
\end{Highlighting}
\end{Shaded}

\begin{verbatim}
## [1] 39.45543
\end{verbatim}

\begin{Shaded}
\begin{Highlighting}[]
\KeywordTok{wilcox.test}\NormalTok{(EPA_Ozone_}\FloatTok{2017.}\NormalTok{data.Processed}\OperatorTok{$}\NormalTok{DAILY_AQI_VALUE,}
\NormalTok{            EPA_Ozone_}\FloatTok{2018.}\NormalTok{data.Processed}\OperatorTok{$}\NormalTok{DAILY_AQI_VALUE, }\DataTypeTok{alternative =} \StringTok{"greater"}\NormalTok{)}
\end{Highlighting}
\end{Shaded}

\begin{verbatim}
## 
##  Wilcoxon rank sum test with continuity correction
## 
## data:  EPA_Ozone_2017.data.Processed$DAILY_AQI_VALUE and EPA_Ozone_2018.data.Processed$DAILY_AQI_VALUE
## W = 58178000, p-value = 9.077e-13
## alternative hypothesis: true location shift is greater than 0
\end{verbatim}

\begin{Shaded}
\begin{Highlighting}[]
\KeywordTok{wilcox.test}\NormalTok{(month_1_}\DecValTok{2017}\OperatorTok{$}\NormalTok{DAILY_AQI_VALUE,month_1_}\DecValTok{2018}\OperatorTok{$}\NormalTok{DAILY_AQI_VALUE,}
            \DataTypeTok{alternative =} \StringTok{"greater"}\NormalTok{)}
\end{Highlighting}
\end{Shaded}

\begin{verbatim}
## 
##  Wilcoxon rank sum test with continuity correction
## 
## data:  month_1_2017$DAILY_AQI_VALUE and month_1_2018$DAILY_AQI_VALUE
## W = 30854, p-value = 1
## alternative hypothesis: true location shift is greater than 0
\end{verbatim}

\begin{Shaded}
\begin{Highlighting}[]
\KeywordTok{wilcox.test}\NormalTok{(month_2_}\DecValTok{2017}\OperatorTok{$}\NormalTok{DAILY_AQI_VALUE,month_2_}\DecValTok{2018}\OperatorTok{$}\NormalTok{DAILY_AQI_VALUE, }
            \DataTypeTok{alternative =} \StringTok{"greater"}\NormalTok{)}
\end{Highlighting}
\end{Shaded}

\begin{verbatim}
## 
##  Wilcoxon rank sum test with continuity correction
## 
## data:  month_2_2017$DAILY_AQI_VALUE and month_2_2018$DAILY_AQI_VALUE
## W = 114220, p-value < 2.2e-16
## alternative hypothesis: true location shift is greater than 0
\end{verbatim}

\begin{Shaded}
\begin{Highlighting}[]
\KeywordTok{wilcox.test}\NormalTok{(month_3_}\DecValTok{2017}\OperatorTok{$}\NormalTok{DAILY_AQI_VALUE,month_3_}\DecValTok{2018}\OperatorTok{$}\NormalTok{DAILY_AQI_VALUE, }
            \DataTypeTok{alternative =} \StringTok{"greater"}\NormalTok{)}
\end{Highlighting}
\end{Shaded}

\begin{verbatim}
## 
##  Wilcoxon rank sum test with continuity correction
## 
## data:  month_3_2017$DAILY_AQI_VALUE and month_3_2018$DAILY_AQI_VALUE
## W = 725790, p-value = 0.0865
## alternative hypothesis: true location shift is greater than 0
\end{verbatim}

\begin{Shaded}
\begin{Highlighting}[]
\KeywordTok{wilcox.test}\NormalTok{(month_4_}\DecValTok{2017}\OperatorTok{$}\NormalTok{DAILY_AQI_VALUE,month_4_}\DecValTok{2018}\OperatorTok{$}\NormalTok{DAILY_AQI_VALUE, }
            \DataTypeTok{alternative =} \StringTok{"greater"}\NormalTok{)}
\end{Highlighting}
\end{Shaded}

\begin{verbatim}
## 
##  Wilcoxon rank sum test with continuity correction
## 
## data:  month_4_2017$DAILY_AQI_VALUE and month_4_2018$DAILY_AQI_VALUE
## W = 486820, p-value = 1
## alternative hypothesis: true location shift is greater than 0
\end{verbatim}

\begin{Shaded}
\begin{Highlighting}[]
\KeywordTok{wilcox.test}\NormalTok{(month_5_}\DecValTok{2017}\OperatorTok{$}\NormalTok{DAILY_AQI_VALUE,month_5_}\DecValTok{2018}\OperatorTok{$}\NormalTok{DAILY_AQI_VALUE, }
            \DataTypeTok{alternative =} \StringTok{"greater"}\NormalTok{)}
\end{Highlighting}
\end{Shaded}

\begin{verbatim}
## 
##  Wilcoxon rank sum test with continuity correction
## 
## data:  month_5_2017$DAILY_AQI_VALUE and month_5_2018$DAILY_AQI_VALUE
## W = 794380, p-value = 1.553e-07
## alternative hypothesis: true location shift is greater than 0
\end{verbatim}

\begin{Shaded}
\begin{Highlighting}[]
\KeywordTok{wilcox.test}\NormalTok{(month_6_}\DecValTok{2017}\OperatorTok{$}\NormalTok{DAILY_AQI_VALUE,month_6_}\DecValTok{2018}\OperatorTok{$}\NormalTok{DAILY_AQI_VALUE, }
            \DataTypeTok{alternative =} \StringTok{"greater"}\NormalTok{)}
\end{Highlighting}
\end{Shaded}

\begin{verbatim}
## 
##  Wilcoxon rank sum test with continuity correction
## 
## data:  month_6_2017$DAILY_AQI_VALUE and month_6_2018$DAILY_AQI_VALUE
## W = 503450, p-value = 1
## alternative hypothesis: true location shift is greater than 0
\end{verbatim}

\begin{Shaded}
\begin{Highlighting}[]
\KeywordTok{wilcox.test}\NormalTok{(month_7_}\DecValTok{2017}\OperatorTok{$}\NormalTok{DAILY_AQI_VALUE,month_7_}\DecValTok{2018}\OperatorTok{$}\NormalTok{DAILY_AQI_VALUE, }
            \DataTypeTok{alternative =} \StringTok{"greater"}\NormalTok{)}
\end{Highlighting}
\end{Shaded}

\begin{verbatim}
## 
##  Wilcoxon rank sum test with continuity correction
## 
## data:  month_7_2017$DAILY_AQI_VALUE and month_7_2018$DAILY_AQI_VALUE
## W = 689380, p-value = 0.6059
## alternative hypothesis: true location shift is greater than 0
\end{verbatim}

\begin{Shaded}
\begin{Highlighting}[]
\KeywordTok{wilcox.test}\NormalTok{(month_8_}\DecValTok{2017}\OperatorTok{$}\NormalTok{DAILY_AQI_VALUE,month_8_}\DecValTok{2018}\OperatorTok{$}\NormalTok{DAILY_AQI_VALUE, }
            \DataTypeTok{alternative =} \StringTok{"greater"}\NormalTok{)}
\end{Highlighting}
\end{Shaded}

\begin{verbatim}
## 
##  Wilcoxon rank sum test with continuity correction
## 
## data:  month_8_2017$DAILY_AQI_VALUE and month_8_2018$DAILY_AQI_VALUE
## W = 767460, p-value = 7.514e-07
## alternative hypothesis: true location shift is greater than 0
\end{verbatim}

\begin{Shaded}
\begin{Highlighting}[]
\KeywordTok{wilcox.test}\NormalTok{(month_9_}\DecValTok{2017}\OperatorTok{$}\NormalTok{DAILY_AQI_VALUE,month_9_}\DecValTok{2018}\OperatorTok{$}\NormalTok{DAILY_AQI_VALUE, }
            \DataTypeTok{alternative =} \StringTok{"greater"}\NormalTok{)}
\end{Highlighting}
\end{Shaded}

\begin{verbatim}
## 
##  Wilcoxon rank sum test with continuity correction
## 
## data:  month_9_2017$DAILY_AQI_VALUE and month_9_2018$DAILY_AQI_VALUE
## W = 618390, p-value < 2.2e-16
## alternative hypothesis: true location shift is greater than 0
\end{verbatim}

\begin{Shaded}
\begin{Highlighting}[]
\KeywordTok{wilcox.test}\NormalTok{(month_10_}\DecValTok{2017}\OperatorTok{$}\NormalTok{DAILY_AQI_VALUE,}
\NormalTok{            month_10_}\DecValTok{2018}\OperatorTok{$}\NormalTok{DAILY_AQI_VALUE, }\DataTypeTok{alternative =} \StringTok{"greater"}\NormalTok{)}
\end{Highlighting}
\end{Shaded}

\begin{verbatim}
## 
##  Wilcoxon rank sum test with continuity correction
## 
## data:  month_10_2017$DAILY_AQI_VALUE and month_10_2018$DAILY_AQI_VALUE
## W = 607880, p-value = 0.03632
## alternative hypothesis: true location shift is greater than 0
\end{verbatim}

\begin{Shaded}
\begin{Highlighting}[]
\KeywordTok{wilcox.test}\NormalTok{(month_11_}\DecValTok{2017}\OperatorTok{$}\NormalTok{DAILY_AQI_VALUE,}
\NormalTok{            month_11_}\DecValTok{2018}\OperatorTok{$}\NormalTok{DAILY_AQI_VALUE, }\DataTypeTok{alternative =} \StringTok{"greater"}\NormalTok{)}
\end{Highlighting}
\end{Shaded}

\begin{verbatim}
## 
##  Wilcoxon rank sum test with continuity correction
## 
## data:  month_11_2017$DAILY_AQI_VALUE and month_11_2018$DAILY_AQI_VALUE
## W = 89283, p-value = 0.0003441
## alternative hypothesis: true location shift is greater than 0
\end{verbatim}

\begin{Shaded}
\begin{Highlighting}[]
\KeywordTok{wilcox.test}\NormalTok{(month_12_}\DecValTok{2017}\OperatorTok{$}\NormalTok{DAILY_AQI_VALUE,}
\NormalTok{            month_12_}\DecValTok{2018}\OperatorTok{$}\NormalTok{DAILY_AQI_VALUE, }\DataTypeTok{alternative =} \StringTok{"greater"}\NormalTok{)}
\end{Highlighting}
\end{Shaded}

\begin{verbatim}
## 
##  Wilcoxon rank sum test with continuity correction
## 
## data:  month_12_2017$DAILY_AQI_VALUE and month_12_2018$DAILY_AQI_VALUE
## W = 25216, p-value = 0.0003636
## alternative hypothesis: true location shift is greater than 0
\end{verbatim}

\begin{figure}
\centering
\includegraphics{Xia_ENV_872_Project_files/figure-latex/Final visualization 1-1.pdf}
\caption{2017 and 2018 ozone compare by year and by month}
\end{figure}

\begin{Shaded}
\begin{Highlighting}[]
\CommentTok{#Mann-Kendall test for trend of Ozone in both 2017 and 2018.}
\CommentTok{#group_by date}
\NormalTok{Ozonebydate_}\DecValTok{2017}\NormalTok{ <-}\StringTok{ }\NormalTok{EPA_Ozone_}\FloatTok{2017.}\NormalTok{data.Processed }\OperatorTok\StringTok{ }
\StringTok{   }\KeywordTok{group_by}\NormalTok{(Date) }\OperatorTok\StringTok{ }
\StringTok{  }\KeywordTok{summarise}\NormalTok{(}\DataTypeTok{MeanOzoneAQI =} \KeywordTok{mean}\NormalTok{(DAILY_AQI_VALUE))}

\NormalTok{Ozonebydate_}\DecValTok{2018}\NormalTok{ <-}\StringTok{ }\NormalTok{EPA_Ozone_}\FloatTok{2018.}\NormalTok{data.Processed }\OperatorTok\StringTok{ }
\StringTok{   }\KeywordTok{group_by}\NormalTok{(Date) }\OperatorTok\StringTok{ }
\StringTok{  }\KeywordTok{summarise}\NormalTok{(}\DataTypeTok{MeanOzoneAQI =} \KeywordTok{mean}\NormalTok{(DAILY_AQI_VALUE))}
\CommentTok{#Non-normal data 2017:}

\CommentTok{#Test for Change point}
\KeywordTok{pettitt.test}\NormalTok{(Ozonebydate_}\DecValTok{2017}\OperatorTok{$}\NormalTok{MeanOzoneAQI)}
\end{Highlighting}
\end{Shaded}

\begin{verbatim}
## 
##  Pettitt's test for single change-point detection
## 
## data:  Ozonebydate_2017$MeanOzoneAQI
## U* = 14903, p-value = 2.158e-12
## alternative hypothesis: two.sided
## sample estimates:
## probable change point at time K 
##                             218
\end{verbatim}

\begin{Shaded}
\begin{Highlighting}[]
\CommentTok{# 218: 2017-08-06}

\CommentTok{# Run seperate Mann-Kendall for each change point: possitive trend}
  \KeywordTok{mk.test}\NormalTok{(Ozonebydate_}\DecValTok{2017}\OperatorTok{$}\NormalTok{MeanOzoneAQI[}\DecValTok{1}\OperatorTok{:}\DecValTok{218}\NormalTok{])}
\end{Highlighting}
\end{Shaded}

\begin{verbatim}
## 
##  Mann-Kendall trend test
## 
## data:  Ozonebydate_2017$MeanOzoneAQI[1:218]
## z = 4.6491, n = 218, p-value = 3.334e-06
## alternative hypothesis: true S is not equal to 0
## sample estimates:
##            S         varS          tau 
## 5.006000e+03 1.158979e+06 2.117105e-01
\end{verbatim}

\begin{Shaded}
\begin{Highlighting}[]
  \CommentTok{#negative trend}
\KeywordTok{mk.test}\NormalTok{(Ozonebydate_}\DecValTok{2017}\OperatorTok{$}\NormalTok{MeanOzoneAQI[}\DecValTok{218}\OperatorTok{:}\DecValTok{364}\NormalTok{])}
\end{Highlighting}
\end{Shaded}

\begin{verbatim}
## 
##  Mann-Kendall trend test
## 
## data:  Ozonebydate_2017$MeanOzoneAQI[218:364]
## z = -4.2373, n = 147, p-value = 2.262e-05
## alternative hypothesis: true S is not equal to 0
## sample estimates:
##             S          varS           tau 
## -2.531000e+03  3.564963e+05 -2.359687e-01
\end{verbatim}

\begin{Shaded}
\begin{Highlighting}[]
\CommentTok{#Another Change point at 280 : 2017-10-07}
\KeywordTok{pettitt.test}\NormalTok{(Ozonebydate_}\DecValTok{2017}\OperatorTok{$}\NormalTok{MeanOzoneAQI[}\DecValTok{218}\OperatorTok{:}\DecValTok{364}\NormalTok{]) }
\end{Highlighting}
\end{Shaded}

\begin{verbatim}
## 
##  Pettitt's test for single change-point detection
## 
## data:  Ozonebydate_2017$MeanOzoneAQI[218:364]
## U* = 2666, p-value = 3.236e-06
## alternative hypothesis: two.sided
## sample estimates:
## probable change point at time K 
##                              62
\end{verbatim}

\begin{Shaded}
\begin{Highlighting}[]
\KeywordTok{mk.test}\NormalTok{(Ozonebydate_}\DecValTok{2017}\OperatorTok{$}\NormalTok{MeanOzoneAQI[}\DecValTok{218}\OperatorTok{:}\DecValTok{280}\NormalTok{]) }\CommentTok{#no significant trend}
\end{Highlighting}
\end{Shaded}

\begin{verbatim}
## 
##  Mann-Kendall trend test
## 
## data:  Ozonebydate_2017$MeanOzoneAQI[218:280]
## z = 1.8031, n = 63, p-value = 0.07138
## alternative hypothesis: true S is not equal to 0
## sample estimates:
##          S       varS        tau 
## 3.0500e+02 2.8427e+04 1.5617e-01
\end{verbatim}

\begin{Shaded}
\begin{Highlighting}[]
\KeywordTok{mk.test}\NormalTok{(Ozonebydate_}\DecValTok{2017}\OperatorTok{$}\NormalTok{MeanOzoneAQI[}\DecValTok{281}\OperatorTok{:}\DecValTok{364}\NormalTok{]) }\CommentTok{#no significant trend}
\end{Highlighting}
\end{Shaded}

\begin{verbatim}
## 
##  Mann-Kendall trend test
## 
## data:  Ozonebydate_2017$MeanOzoneAQI[281:364]
## z = -1.1551, n = 84, p-value = 0.248
## alternative hypothesis: true S is not equal to 0
## sample estimates:
##             S          varS           tau 
## -3.000000e+02  6.699933e+04 -8.615744e-02
\end{verbatim}

\begin{Shaded}
\begin{Highlighting}[]
\CommentTok{#anythird change point?}
\KeywordTok{pettitt.test}\NormalTok{(Ozonebydate_}\DecValTok{2017}\OperatorTok{$}\NormalTok{MeanOzoneAQI[}\DecValTok{281}\OperatorTok{:}\DecValTok{364}\NormalTok{]) }\CommentTok{#not significant}
\end{Highlighting}
\end{Shaded}

\begin{verbatim}
## 
##  Pettitt's test for single change-point detection
## 
## data:  Ozonebydate_2017$MeanOzoneAQI[281:364]
## U* = 566, p-value = 0.08113
## alternative hypothesis: two.sided
## sample estimates:
## probable change point at time K 
##                              59
\end{verbatim}

\begin{Shaded}
\begin{Highlighting}[]
\CommentTok{#Non-normal data 2018:}

\CommentTok{#Test for Change point}
\KeywordTok{pettitt.test}\NormalTok{(Ozonebydate_}\DecValTok{2018}\OperatorTok{$}\NormalTok{MeanOzoneAQI)}
\end{Highlighting}
\end{Shaded}

\begin{verbatim}
## 
##  Pettitt's test for single change-point detection
## 
## data:  Ozonebydate_2018$MeanOzoneAQI
## U* = 14869, p-value = 1.165e-14
## alternative hypothesis: two.sided
## sample estimates:
## probable change point at time K 
##                             210
\end{verbatim}

\begin{Shaded}
\begin{Highlighting}[]
\CommentTok{# 210: 2018-07-29}

\CommentTok{# Run seperate Mann-Kendall for each change point: possitive trend}
  \KeywordTok{mk.test}\NormalTok{(Ozonebydate_}\DecValTok{2018}\OperatorTok{$}\NormalTok{MeanOzoneAQI[}\DecValTok{1}\OperatorTok{:}\DecValTok{210}\NormalTok{])}
\end{Highlighting}
\end{Shaded}

\begin{verbatim}
## 
##  Mann-Kendall trend test
## 
## data:  Ozonebydate_2018$MeanOzoneAQI[1:210]
## z = 5.6612, n = 210, p-value = 1.503e-08
## alternative hypothesis: true S is not equal to 0
## sample estimates:
##            S         varS          tau 
## 5.764000e+03 1.036289e+06 2.626746e-01
\end{verbatim}

\begin{Shaded}
\begin{Highlighting}[]
  \CommentTok{#negative trend}
\KeywordTok{mk.test}\NormalTok{(Ozonebydate_}\DecValTok{2018}\OperatorTok{$}\NormalTok{MeanOzoneAQI[}\DecValTok{210}\OperatorTok{:}\DecValTok{343}\NormalTok{])}
\end{Highlighting}
\end{Shaded}

\begin{verbatim}
## 
##  Mann-Kendall trend test
## 
## data:  Ozonebydate_2018$MeanOzoneAQI[210:343]
## z = -3.8546, n = 134, p-value = 0.0001159
## alternative hypothesis: true S is not equal to 0
## sample estimates:
##             S          varS           tau 
## -2.005000e+03  2.702983e+05 -2.250281e-01
\end{verbatim}

\begin{Shaded}
\begin{Highlighting}[]
\CommentTok{#Another Change point at 252 : 2017-09-09}
\KeywordTok{pettitt.test}\NormalTok{(Ozonebydate_}\DecValTok{2018}\OperatorTok{$}\NormalTok{MeanOzoneAQI[}\DecValTok{210}\OperatorTok{:}\DecValTok{343}\NormalTok{]) }
\end{Highlighting}
\end{Shaded}

\begin{verbatim}
## 
##  Pettitt's test for single change-point detection
## 
## data:  Ozonebydate_2018$MeanOzoneAQI[210:343]
## U* = 1957, p-value = 0.0001528
## alternative hypothesis: two.sided
## sample estimates:
## probable change point at time K 
##                              42
\end{verbatim}

\begin{Shaded}
\begin{Highlighting}[]
\KeywordTok{mk.test}\NormalTok{(Ozonebydate_}\DecValTok{2018}\OperatorTok{$}\NormalTok{MeanOzoneAQI[}\DecValTok{210}\OperatorTok{:}\DecValTok{252}\NormalTok{]) }\CommentTok{#no significant trend}
\end{Highlighting}
\end{Shaded}

\begin{verbatim}
## 
##  Mann-Kendall trend test
## 
## data:  Ozonebydate_2018$MeanOzoneAQI[210:252]
## z = 1.3605, n = 43, p-value = 0.1737
## alternative hypothesis: true S is not equal to 0
## sample estimates:
##           S        varS         tau 
##  131.000000 9130.333333    0.145072
\end{verbatim}

\begin{Shaded}
\begin{Highlighting}[]
\KeywordTok{mk.test}\NormalTok{(Ozonebydate_}\DecValTok{2018}\OperatorTok{$}\NormalTok{MeanOzoneAQI[}\DecValTok{252}\OperatorTok{:}\DecValTok{343}\NormalTok{]) }\CommentTok{#no significant trend}
\end{Highlighting}
\end{Shaded}

\begin{verbatim}
## 
##  Mann-Kendall trend test
## 
## data:  Ozonebydate_2018$MeanOzoneAQI[252:343]
## z = -0.74202, n = 92, p-value = 0.4581
## alternative hypothesis: true S is not equal to 0
## sample estimates:
##             S          varS           tau 
## -2.210000e+02  8.790500e+04 -5.280134e-02
\end{verbatim}

\begin{Shaded}
\begin{Highlighting}[]
\CommentTok{#anythird change point? }
\KeywordTok{pettitt.test}\NormalTok{(Ozonebydate_}\DecValTok{2017}\OperatorTok{$}\NormalTok{MeanOzoneAQI[}\DecValTok{252}\OperatorTok{:}\DecValTok{343}\NormalTok{]) }\CommentTok{#At 280, 2018-10-07}
\end{Highlighting}
\end{Shaded}

\begin{verbatim}
## 
##  Pettitt's test for single change-point detection
## 
## data:  Ozonebydate_2017$MeanOzoneAQI[252:343]
## U* = 938, p-value = 0.002446
## alternative hypothesis: two.sided
## sample estimates:
## probable change point at time K 
##                              28
\end{verbatim}

\begin{Shaded}
\begin{Highlighting}[]
\KeywordTok{mk.test}\NormalTok{(Ozonebydate_}\DecValTok{2018}\OperatorTok{$}\NormalTok{MeanOzoneAQI[}\DecValTok{252}\OperatorTok{:}\DecValTok{280}\NormalTok{])}
\end{Highlighting}
\end{Shaded}

\begin{verbatim}
## 
##  Mann-Kendall trend test
## 
## data:  Ozonebydate_2018$MeanOzoneAQI[252:280]
## z = 2.6824, n = 29, p-value = 0.00731
## alternative hypothesis: true S is not equal to 0
## sample estimates:
##            S         varS          tau 
##  144.0000000 2842.0000000    0.3546798
\end{verbatim}

\begin{Shaded}
\begin{Highlighting}[]
\KeywordTok{mk.test}\NormalTok{(Ozonebydate_}\DecValTok{2018}\OperatorTok{$}\NormalTok{MeanOzoneAQI[}\DecValTok{280}\OperatorTok{:}\DecValTok{343}\NormalTok{])}
\end{Highlighting}
\end{Shaded}

\begin{verbatim}
## 
##  Mann-Kendall trend test
## 
## data:  Ozonebydate_2018$MeanOzoneAQI[280:343]
## z = -2.2827, n = 64, p-value = 0.02245
## alternative hypothesis: true S is not equal to 0
## sample estimates:
##             S          varS           tau 
##  -395.0000000 29791.0000000    -0.1959812
\end{verbatim}

\pagebreak
\includegraphics{Xia_ENV_872_Project_files/figure-latex/Final visualization 2-1.pdf}

\begin{figure}
\centering
\includegraphics{Xia_ENV_872_Project_files/figure-latex/Final visualization 3-1.pdf}
\caption{2018 ozone trend}
\end{figure}

\begin{Shaded}
\begin{Highlighting}[]
\CommentTok{#Logistic Regression}
\end{Highlighting}
\end{Shaded}

\newpage

\section{Summary and Conclusions}\label{summary-and-conclusions}


\end{document}
